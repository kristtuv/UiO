\documentclass[a4paper, 12pt]{article}
\usepackage[utf8]{inputenc}
\usepackage[T1]{fontenc}
\usepackage{textcomp, color, amsmath, amssymb, tikz, subfig, float, mathrsfs}
\usepackage{xcolor}
\usepackage{amsfonts}
\usepackage{graphicx}
\usepackage{listings}
\usepackage{ragged2e}
\usepackage{amsmath}
\usepackage[export]{adjustbox}
\usepackage[]{esint}
\usepackage{hyperref}
\usepackage[skins,theorems]{tcolorbox}
\usepackage{cite}
\usepackage{algorithm}
\usepackage{algorithmicx}
\usepackage{algpseudocode}
\usepackage{subfig}% http://ctan.org/pkg/subfig
\usepackage{hyperref}
\usepackage{setspace}
\newsubfloat{figure}
%Tikzcommands
\usepackage{tikz}
\usetikzlibrary{shapes.geometric, arrows}
\tikzstyle{startstop} = [rectangle, rounded corners, minimum width=3cm, minimum height=1cm,text centered, draw=black, fill=red!30]
\tikzstyle{io} = [trapezium, trapezium left angle=70, trapezium right angle=110, minimum width=3cm, minimum height=1cm, text centered, draw=black, fill=blue!30]
\tikzstyle{process} = [rectangle, minimum width=3cm, minimum height=1cm, text centered, draw=black, fill=orange!30]
\tikzstyle{decision} = [diamond, minimum width=3cm, minimum height=1cm, text centered, draw=black, fill=green!30]
\tikzstyle{arrow} = [thick,->,>=stealth]




\tcbset{highlight math style={enhanced,
  colframe=red,colback=white,arc=10pt,boxrule=0.5pt, hbox}}

\definecolor{lightgray}{gray}{0.75}

\newcommand\greybox[1]{%
  \vskip\baselineskip%
  \par\noindent\colorbox{lightgray}{%
    \begin{minipage}{\textwidth}#1\end{minipage}%
  }%
  \vskip\baselineskip%
}
	
\newcommand{\vertfig}[2][]{%
  \begin{minipage}{5in}\subfloat[#1]{#2}\end{minipage}}
  
\newcommand{\horizfig}[2][]{%
  \begin{minipage}{3in}\subfloat[#1]{#2}\end{minipage}}

\linespread{1.5}
\begin{document}



\author{Kristian Tuv}
\title{FYS3710}
\maketitle

\newpage
\tableofcontents
\newpage

\section{Ytre membran}
Den ytre membranen i mitokondiret omslutter hele organellen og er omtrentlig 6 - 7.5 nm tykk. Den har en oppbygging som er omtrentlig lik på en eukaryot celle med proteiner og phospholipider.\\

Membranen har et stort antall integrale membran proteiner som kalles poriner
\cite{tutorpace}. Porinene fungerer som porer i membranen og er ansvarlige for transport av små molekyler på størrelse 5000 daltons eller mindre \cite{tutorvista}.\\

Større proteiner må ha den rette signalfrekvensen for å slippe igjennom membranen.\cite{Neupert}. Når proteinene frigjøres fra ribosomene i cytosolen inneholder de en signalfrekvens som gjør til at de kan binde seg til et  transport protein. Disse transport proteinene, som med en samlebetegnelse kalles for translocase\cite{medic} proteiner, er en protein subenhet bestående av kun ett proteinmolekyl\cite{subunit}. Dersom signalfrekvensen er korrekt, frakter translokaseproteinene de større proteinene gjennom porer i membranen og inn i det intermembrane rommet. \\

\section{Det intermembrane rommet}
Dette er et lite rom som befinner seg mellom den ytre og den indre membranen. På grunn av den relativt frie passasjen av små molekyler som ioner og sukker fra cytosolen, er innholdet i det intermembrane rommet lignende på cytosolen. Proteininnholdet vil derimot være ganske forskjellig siden det kreves en signalfrekvens for at større molekyler skal slippe igjennom. \cite{Herrmann}

\begin{thebibliography}{1}

\bibitem{tutorpace}
 \url{ http://biology.tutorpace.com/mitochondria-online-tutoring}, 03.09.2017
\bibitem{tutorvista}
\url{http://biology.tutorvista.com/animal-and-plant-cells/mitochondria.html}, 03.09.2017,

\bibitem{medic}
 \url{http://medical-dictionary.thefreedictionary.com/translocase}, 04.09.2017, 
 
\bibitem{subunit}
\url{https://en.wikipedia.org/wiki/Protein_subunit}

\bibitem{Neupert}
Walter Neupert, Johannes M.Herrmann, Vol. 76:723-749, 30.01.2007.
 \url{http://www.annualreviews.org/doi/10.1146/annurev.biochem.76.052705.163409}
 
 \bibitem{Herrmann}
  Johannes M. Herrmann, Jan Riemer, Anitoxidant and redox signaling, volume 13, number 9, 2010, p.1342
  \url{http://online.liebertpub.com/doi/pdf/10.1089/ars.2009.3063}

\end{thebibliography}

\end{document} 